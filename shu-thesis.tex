% UTF-8 encoding
\documentclass[UTF8]{ctexart}
%====================================================
% 颜色和自定义颜色
\usepackage{xcolor}
\definecolor{hyperref-green}{RGB}{0,150,0}
\definecolor{hyperref-red}{RGB}{200,0,0}
\definecolor{hyperref-blue}{RGB}{0,0,200}
%====================================================

%====================================================
% 超链接
\usepackage[
pagebackref=false,  %这里关闭参考文献的反向超链接
citecolor=black,
linkcolor=black,
urlcolor=hyperref-blue,
menucolor=black,
letterpaper=true,
breaklinks=true,
bookmarks=true,
colorlinks
]{hyperref}
\usepackage{bibentry}
%====================================================

%====================================================
%导入外部pdf
\usepackage{pdfpages} % includepdf
%====================================================

%====================================================
% 表格,woo
\usepackage{booktabs}
\usepackage{multirow}
\usepackage{makecell}
%====================================================

%====================================================
% 公式和数学标识,算法流程图
\usepackage{amsmath}
\usepackage{bm} %%某些矢量需要加粗字符,除mathbf外的另一种方式%By Kuber
\usepackage{amssymb}
\usepackage{pifont}
\usepackage{listings} %插入代码
\usepackage[ruled]{algorithm2e} %算法和伪代码
\newcommand{\cmark}{\checkmark}%
\newcommand{\xmark}{\ding{55}}%
\usepackage{xcolor} % math color
\usepackage{diagbox}
\numberwithin{equation}{section} % 公式按章节编号
\numberwithin{table}{section} % 公式按章节编号
% 声明argmin和argmax运算符
\newcommand{\argmin}{\mathop{\mathrm{argmin}}\limits}
\newcommand{\argmax}{\mathop{\mathrm{argmax}}\limits}
\usepackage{amsfonts}
%====================================================

%====================================================
% 插入图表和图表描述
\usepackage{graphicx, caption, subfigure, float}
\DeclareCaptionFormat{myformat}{\fontsize{11}{8}\selectfont#1#2#3}  %这里修改图片标题字体大小
\captionsetup{format=myformat}
% \captionsetup{margin=2cm}
\captionsetup{justification=centering}
\usepackage{overpic}
% 绘图
\usepackage{pgfplots}
%====================================================

%====================================================
% 其他格式,和样式
% 将章节标号转为中文
\usepackage{zhnumber}  
%% geometry
\usepackage{geometry}
\geometry{left=3.17cm,right=3.17cm,top=3.0cm,bottom=3.0cm}  % 页边距
% 页眉相关的设置
\usepackage{fancyhdr}
\fancyhf{}  % 清除默认页眉
\pagestyle{fancy}
\lhead{上海大学硕士学位论文}  % 添加右侧页眉
\cfoot{\thepage}  % 添加页脚页码
%% 设置章节格式
\CTEXsetup[name={第, 章}]{section}
\CTEXsetup[number={\chinese{section}}]{section}
\CTEXsetup[format+={\zihao{-2}}]{section}   % 大章节字体: 小二
\CTEXsetup[beforeskip={17pt}]{section} % 标题与上面文本的间距%By Kuber
\CTEXsetup[afterskip={16.5pt}]{section}  % 标题与下面文本的间距%By Kuber
\CTEXsetup[format+={\zihao{3}}]{subsection} % 小章节字体: 三号
\CTEXsetup[beforeskip={13pt}]{subsection}%同上,间距可调
\CTEXsetup[afterskip={13pt}]{subsection}%同上,间距可调
\CTEXsetup[beforeskip={13pt}]{subsubsection}%同上,间距可调
\CTEXsetup[afterskip={13pt}]{subsubsection}%同上,间距可调
\CTEXsetup[format+={\zihao{3}}]{subsection} % 小章节字体: 三号
\CTEXsetup[format+={\zihao{-4}}]{subsubsection}
% 图表按章节编号
\usepackage{chngcntr}
\counterwithin{figure}{section}

\usepackage[shortlabels]{enumitem}

%====================================================

%====================================================
%常用的命令
\newcommand{\red}[1]{{\textcolor{red}{#1}}}%
%====================================================

%====================================================
% 开始正文
\begin{document}
\zihao{-4} %
\linespread{1.6} \selectfont  % 调整全文为1.6倍line间距,非常接近word版本1.5行间距%By Kuber

% 导入封面内容,注意这个地方的页码请根据你的实际情况设置,
\includepdf[pages={1-7}]{cover.pdf}
\pagenumbering{Roman} % 目录之前的内容(包括目录)页码使用罗马数字
\setcounter{page}{7}  % LaTeX的起始页码

% 由于目录(TOC: table of contents)也会被hyperref作为超链接,因此颜色会被设置为和图表超链接一样的红色,红色的目录不好看,这里单独把TOC的颜色设置为黑色。
{\hypersetup{linkcolor=black}
\tableofcontents}
\pagebreak
{\hypersetup{linkcolor=black}
\renewcommand\listfigurename{图\;目\;录}
\listoffigures}
\pagebreak
{\hypersetup{linkcolor=black}
\renewcommand\listtablename{表\;目\;录}
\listoftables}

\pagebreak
%%%%%%%%%%%%%%%%%%%%%%%%%%%%%%%%%%%%%%%%%%%%%%%%%%%%%%%
\section{第一章标题}
%%%%%%%%%%%%%%%%%%%%%%%%%%%%%%%%%%%%%%%%%%%%%%%%%%%%%%%
\pagenumbering{arabic} % 正文开始,页码使用阿拉伯数字
\setcounter{page}{1}

\subsection{课题来源}
这是一篇参考文献\cite{zheng2013personalized}。
\subsection{国内外研究现状与问题分析}

\subsubsection{xxx}

\subsubsection{xxx}

\subsection{本文章节安排}

\newpage

% %%%%%%%%%%%%%%%%%%%%%%%%%%%%%%%%%%%%%%%%%%%%%%%%%%%%%%%
\section{第二章标题}
% %%%%%%%%%%%%%%%%%%%%%%%%%%%%%%%%%%%%%%%%%%%%%%%%%%%%%%%

\newpage
% %%%%%%%%%%%%%%%%%%%%%%%%%%%%%%%%%%%%%%%%%%%%%%%%%%%%%%%
\section{第三章标题}
% %%%%%%%%%%%%%%%%%%%%%%%%%%%%%%%%%%%%%%%%%%%%%%%%%%%%%%%

\newpage
% %%%%%%%%%%%%%%%%%%%%%%%%%%%%%%%%%%%%%%%%%%%%%%%%%%%%%%%
\section{第四章标题}
% %%%%%%%%%%%%%%%%%%%%%%%%%%%%%%%%%%%%%%%%%%%%%%%%%%%%%%%

\newpage
% %%%%%%%%%%%%%%%%%%%%%%%%%%%%%%%%%%%%%%%%%%%%%%%%%%%%%%%
\section{第五章标题}
% %%%%%%%%%%%%%%%%%%%%%%%%%%%%%%%%%%%%%%%%%%%%%%%%%%%%%%%

\newpage
% %%%%%%%%%%%%%%%%%%%%%%%%%%%%%%%%%%%%%%%%%%%%%%%%%%%%%%%
\section{第六章标题}
% %%%%%%%%%%%%%%%%%%%%%%%%%%%%%%%%%%%%%%%%%%%%%%%%%%%%%%%

\newpage

%% 参考文献
\pagebreak
\zihao{5} % 依据上大Word模板,参考文献字号为5号字体%By Kuber
\bibliographystyle{ieeetr}
\addcontentsline{toc}{section}{参考文献}
\bibliography{shu-thesis}

%%%%%%%%%%%%%%%%%%%%%%%%%%%%%%%%%%%%%%%%%%%%%%%%%%%%%%%%%%%%%%%%%%%%%%%%%%%%%%%%%%%
%%%%%%%%%%%%%%%%%%%%%%%%%%%%%%%%%%%%%%%%%%%%%%%%%%%%%%%%%%%%%%%%%%%%%%%%%%%%%%%%%%%
\pagebreak
\zihao{-4} % 字体调整回小四 %By Kuber
\section*{作者在攻读硕士学位期间公开发表的论文}
\addcontentsline{toc}{section}{作者在攻读硕士学位期间公开发表的论文}

\hangafter 1
\hangindent 1.5em
\noindent
\begin{enumerate}
    \item xxx
    \item xxx
\end{enumerate}


%%%%%%%%%%%%%%%%%%%%%%%%%%%%%%%%%%%%%%%%%%%%%%%%%%%%%%%%%%%%%%%%%%%%%%%%%%%%%%%%%%%
%%%%%%%%%%%%%%%%%%%%%%%%%%%%%%%%%%%%%%%%%%%%%%%%%%%%%%%%%%%%%%%%%%%%%%%%%%%%%%%%%%%
\pagebreak
\section*{作者在攻读硕士学位期间所参与的项目}
\addcontentsline{toc}{section}{作者在攻读硕士学位期间所参与的项目}

\begin{enumerate}
    \item xxx
    \item xxx
\end{enumerate}

\pagebreak
\section*{致谢}
\addcontentsline{toc}{section}{致谢}

\end{document}